\documentclass[12pt]{scrartcl}

\usepackage[ngerman]{babel}
\usepackage[utf8]{inputenc}
\usepackage[T1]{fontenc}
\usepackage{csquotes}
\usepackage{graphicx}
\usepackage[onehalfspacing]{setspace}
\usepackage[a4paper, left=2.5cm, right=2.5cm, top=2.5cm, bottom=2.5cm]{geometry}
\usepackage{tikz}
\usepackage{aeguill}
\usepackage{scalefnt}
\usepackage{url}
\usepackage{hyperref}
\usepackage[figure]{hypcap}
\usepackage{fancyhdr}
\usepackage{verbatim}
\usepackage[printonlyused, nohyperlinks]{acronym}

\begin{document}

	\pagestyle{fancy}
	\fancyhead[L]{Chiara Vogt Melian}
	\fancyhead[R]{24.09.2024}
	\pagenumbering{gobble}	
	\renewcommand{\headrulewidth}{0.1 pt}
	\renewcommand{\footrulewidth}{0 pt}
	\section*{Exposé}
	\subsection*{\textit{Vorhersage von Hotlinenutzung mit sensiblen Anrufszenarien}}
	\subsubsection*{Analyse der Telefonhotline zu Gewalt gegen Frauen in Brasilien 'Ligue 180'}
	Die Nachfrage einer öffentlichen Telefonhotline sollte immer gedeckt sein. Insbesondere dann, wenn die Hotline der Anzeige von Gewalttaten dient. Welcher Algorithmus sich gut eignet, um die Anrufquote der Hotline „Ligue 180“ vorherzusagen, soll in dieser Arbeit erforscht werden. Außerdem soll der Bedarf nach speziell geschulten Arbeitskräften vorhergesagt werden. Die untersuchte Telefonhotline dient der Unterstützung von Frauen in Gewaltsituation in Brasilien. Der Datensatz beinhaltet die Anzeigen, die über diese Hotline seit 2014 aufgegeben werden.\\
	Es soll der Aspekt des Anrufzeitpunkts untersucht werden: dazu sollen in einer ersten Analyse passende Time-Prediction-Analysemethoden verglichen und die besten vier herausgearbeitet werden. Durch den Vergleich der Vorhersagedaten mit den Testdaten und deren Abweichungen wird evaluiert, welche Methode die beste Vorhersage für das Anrufverhalten liefert. Das dient dazu, die Auslastung der Hotline berechnen zu können, um die Erreichbarkeit sicherzustellen.\\
	Anschließend werden Cluster herausgearbeitet, die aus dem Anrufverhalten der Personen und deren sozioökonomischen Faktoren entstehen. Dazu wird mittels Fachliteratur erarbeitet, welche Algorithmen besonders geeignet sind, um Nutzer:innengruppen zu identifizieren und welche Schritte dabei notwendig sind. Die Analyse dient der Identifikation von gängigen und möglichen Anrufszenarien und Bedürfnissen der Nutzer:innen, um das Personal zu schulen.\\
	In einem letzten Schritt soll die Analyse um die räumliche Dimension erweitert werden. Dazu wird ausgewertet, welche Orte besonders häufig von der Hotline Gebrauch machen und ob diese zusammenhängen. Als nächstes werden diese möglichen Cluster mit den Clustern des Anrufverhaltens und denen der identifizierten Personengruppen verglichen. Außerdem können die Ergebnisse mit der physisch vorhandenen Infrastruktur in den Regionen verglichen werden, um mögliche Zusammenhänge zu erforschen. 
	

	
\end{document}	